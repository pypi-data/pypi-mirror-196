\documentclass{article}
\usepackage[utf8]{inputenc}
\begin{document}

LitREPL for latex recognizes \texttt{lcode} environments as code and
\texttt{lresult} as result sections. The tag names is currently hardcoded into
the simple parser the tool is using, so we need to additionally introduce it to
the Latex system. Here we do it in a most simple way.

\newenvironment{lcode}{\begin{texttt}}{\end{texttt}}
\newenvironment{lresult}{\begin{texttt}}{\end{texttt}}
\newcommand{\linline}[2]{#2}

Executable section is the text between the \texttt{lcode} begin/end tags.
Putting the cursor on it and typing the \texttt{:LitEval1} executes it in the
background Python interpreter.

\begin{lcode}
W='Hello, World!'
print(W)
\end{lcode}

\texttt{lresult} tags next to the executable section mark the result section.
The output of the executable section will be pasted here. The
original content of the section will be replaced.

\begin{lresult}
Hello, World!
\end{lresult}

Commented \texttt{lresult}/\texttt{lnoresult} tags also marks result sections.
This way we could customise the Latex markup for every particular section.

\begin{lcode}
print("Hi!")
\end{lcode}

%lresult
Hi!
%lnoresult

Additionally, VimREPL for Latex recognises \texttt{linline} tags for which it
prints its first argument and pastes the result in place of the second argument.

\linline{W}{Hello, World!}

\end{document}

